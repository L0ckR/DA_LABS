\documentclass[12pt,a4paper]{article}

% --- Русский язык и кодировки ---
\usepackage[T2A]{fontenc}
\usepackage[utf8]{inputenc}
\usepackage[russian]{babel}

% --- Поля и межстрочные ---
\usepackage{geometry}
\geometry{margin=2cm}
\usepackage{setspace}
\onehalfspacing

% --- Оформление ---
\usepackage{indentfirst}
\usepackage{enumitem}
\setlist{nosep}

% --- Математика ---
\usepackage{amsmath,amssymb,amsthm}

% --- Код ---
\usepackage{listings}
\usepackage{xcolor}
\lstdefinestyle{cppstyle}{
  language=C++,
  basicstyle=\ttfamily\small,
  keywordstyle=\color{blue},
  commentstyle=\color{gray!70},
  stringstyle=\color{green!40!black},
  numbers=left,
  numberstyle=\tiny\color{gray!70},
  stepnumber=1,
  numbersep=8pt,
  tabsize=2,
  showstringspaces=false,
  breaklines=true,
  frame=single
}

% --- Заголовки разделов ---
\usepackage{titlesec}
\titleformat{\section}{\large\bfseries}{\thesection.}{0.5em}{}
\titleformat{\subsection}{\normalsize\bfseries}{\thesubsection.}{0.5em}{}

% --- Гиперссылки ---
\usepackage{hyperref}
\hypersetup{colorlinks=true, linkcolor=black, urlcolor=blue, citecolor=black}
\documentclass[12pt]{article}

\usepackage{fullpage}
\usepackage{multicol,multirow}
\usepackage{tabularx}
\usepackage{ulem}
\usepackage[utf8]{inputenc}
\usepackage[russian]{babel}
\usepackage{indentfirst}
\usepackage{tikz}
\usepackage{pgfplots}

\begin{document}

% Титульный лист
\begin{titlepage}
    \begin{center}
        \vspace*{1cm}
        
        \textbf{ФЕДЕРАЛЬНОЕ ГОСУДАРСТВЕННОЕ ОБРАЗОВАТЕЛЬНОЕ УЧРЕЖДЕНИЕ ВЫСШЕГО ОБРАЗОВАНИЯ} \\
        \textbf{«МОСКОВСКИЙ АВИАЦИОННЫЙ ИНСТИТУТ (НАЦИОНАЛЬНЫЙ ИССЛЕДОВАТЕЛЬСКИЙ УНИВЕРСИТЕТ)»}
        
        \vspace{3cm}
        
        \textbf{\LARGE ОТЧЕТ} \\
        \vspace{0.5cm}
        \textbf{\Large о выполнении лабораторной работы №7} \\
        \vspace{0.5cm}
        \textbf{\Large «Жадные алгоритмы»} \\
        \vspace{1cm}
        
        \textbf{по дисциплине} \\
        \textit{«Дискретный анализ»}
        
        \vfill
        
        \begin{flushleft}
        Выполнил студент группы М8О-308Б-23: \\
        \textbf{Ибрагимов Далгат Магомедалиевич} \\
        \vspace{0.5cm}
        Проверил: \\
        \textbf{Макаров Н.К.}
        \end{flushleft}
        
        \vspace{3cm}
        
        Москва, 2025
    \end{center}
\end{titlepage}


\subsection*{Постановка задачи}

\textbf{Вариант 3: Максимальный треугольник} 

\section*{Цель работы}
Изучение применения жадных алгоритмов для оптимизации вычислений.

\section*{Задание}
Заданы длины N отрезков, необходимо выбрать три таких отрезка, которые образовывали бы треугольник с максимальной площадью.

Формат ввода:
На первой строке находится число N, за которым следует N строк с целыми числами-длинами отрезков.

Формат вывода:
Если никакого треугольника из заданных отрезков составить нельзя – 0, в противном случае на первой строке – площадь треугольника с тремя знаками после запятой, на второй строке – длины трёх отрезков, составляющих этот треугольник. Длины должны быть отсортированы.

\section*{Реализация программы}
В данной лабораторной работе была реализована программа на языке C++ с использованием жадных алгоритмов. Программа включает следующие функции:

\begin{itemize}
    \item \texttt{double calculateTriangleArea} - вычисляет площадь треугольника по формуле Герона.
\end{itemize}

\section*{Листинг кода}
\begin{verbatim}
#include <iostream>
#include <vector>
#include <algorithm>
#include <cmath>
#include <iomanip>

double calculateTriangleArea(double s1, double s2, double s3, double halfSum) 
{
    return sqrt(halfSum * (halfSum - s1) * (halfSum - s2) * (halfSum - s3));
}

int main() 
{
    int count;
    std::cin >> count;
    std::vector<int> elements(count);
    for (int i = 0; i < count; ++i) 
    {
        std::cin >> elements[i];
    }

    if (count < 3) 
    {
        std::cout << 0 << "\n";
        return 0;
    }

    std::sort(elements.rbegin(), elements.rend());

    int index = 0;
    std::cout << std::fixed << std::setprecision(3);

    double maxTriangleArea = -1.0;
    int foundIndex = -1;
    bool isPossible = false;

    while (index < count - 2) 
    {
        if (elements[index] < elements[index + 1] + elements[index + 2]) 
        {
            double side1 = (double)elements[index], side2 = (double)elements[index + 1], side3 = (double)elements[index + 2];
            double halfSum = (side1 + side2 + side3) / 2.0;

            double currentArea = calculateTriangleArea(side1, side2, side3, halfSum);
            
            if (currentArea > maxTriangleArea) 
            {
                maxTriangleArea = currentArea;
                foundIndex = index;
            }
            isPossible = true;
        }
        ++index;
    }

    if (isPossible) 
    {
        std::cout << maxTriangleArea << "\n";
        std::cout << elements[foundIndex + 2] << " " << elements[foundIndex + 1] << " " << elements[foundIndex] << "\n";
    } 
    else 
    {
        std::cout << 0 << "\n";
    }
}
\end{verbatim}

\section*{Описание работы программы}
Программа принимает количество отрезков и их длины в качестве входных данных. После этого она проверяет, возможно ли сформировать треугольник из заданных отрезков. Для этого используется жадный подход, который включает сортировку длин отрезков в порядке убывания и проверку условия существования треугольника: сумма длины двух сторон должна быть больше длины третьей стороны. Программа вычисляет площадь потенциального треугольника с помощью формулы Герона и выбирает максимальную из возможных площадей. Если удается найти хотя бы один треугольник, программа выводит его площадь и длины отрезков, формирующих этот треугольник.

\section*{Дневник отладки}
\begin{enumerate}
    \item Все этапы разработки программы проходили без значительных ошибок.
    \item В процессе тестирования не было выявлено логических ошибок, и программа корректно обрабатывала различные входные данные.
\end{enumerate}

\section*{Результаты тестирования}
Программа успешно продемонстрировала корректные результаты для различных наборов входных данных, включая крайние случаи, такие как минимальные и максимальные значения длины отрезков. Это подтверждает правильность выбранного алгоритма и его эффективность в решении поставленной задачи.

\subsection*{Тест производительности}
Сложность алгоритма O(n log n)
Для оценки производительности алгоритма была проведена серия тестов, сравнивающих время выполнения данного решения с наивным методом. Входные данные были выбраны таким образом, чтобы включать большие значения длины отрезков, что позволяет проанализировать эффективность алгоритма.

\begin{figure}[htbp]
    \centering
    \begin{tikzpicture}
        \begin{axis}[
            xlabel={n ($\times 10^2$)},
            ylabel={time (ms)},
            grid=major,
            xmin=0, xmax=1000,
            ymin=0, ymax=50,
            xtick={0, 200, 400, 600, 800, 1000},
            ytick={0, 10, 20, 30, 40, 50},
            legend style={at={(0.5,-0.2)},anchor=north},
            legend columns=2,
            width=0.8\textwidth,
            height=0.5\textwidth,
            ]
            \addplot[color=blue,mark=*] coordinates {
                (0.09, 0.015)
                (0.9, 0.018)
                (9, 0.034)
                (99, 0.235)
                (999, 3.45)
            };
            \addlegendentry{Dynamic}
            \addplot[color=red,mark=square] coordinates {
                (0.09, 0.025)
                (0.9, 0.045)
                (9,  0.150)
                (99, 4.05)
                (999, 40.67)
            };
            \addlegendentry{Naive}
        \end{axis}
    \end{tikzpicture}
    \label{fig:graph}
\end{figure}


\newpage
\subsection*{Выводы}
В ходе выполнения лабораторной работы была разработана программа, использующая жадный алгоритм для решения задачи выбора отрезков, формирующих треугольник максимальной площади. Эффективность алгоритма была подтверждена через тестирование и сравнение с наивными методами. Результаты показывают, что применение жадного подхода значительно ускоряет процесс вычислений, что делает его предпочтительным для данной задачи.

\end{document}
