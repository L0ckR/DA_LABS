\documentclass[12pt]{article}

\usepackage{fullpage}
\usepackage{multicol,multirow}
\usepackage{tabularx}
\usepackage{ulem}
\usepackage[utf8]{inputenc}
\usepackage[russian]{babel}
\usepackage{indentfirst}

\usepackage{tikz}
\usepackage{pgfplots}

\begin{document}

\section*{Лабораторная работа №\,1 по курсу дискрeтного анализа: сортировка за линейное время}

\noindent Выполнил студент группы 08-208 МАИ \textit{Ибрагимов Далгат}.

\subsection*{Условие}

\begin{itemize}
\item \textbf{Общая постановка задачи:} Требуется разработать программу, осуществляющую ввод пар «ключ-значение», их упорядочивание по возрастанию ключа указанным алгоритмом сортировки за линейное время и вывод отсортированной последовательности.

\item \textbf{Вариант сортировки:} Сортировка подсчетом

\item \textbf{Вариант ключа:} Числа от 0 до 65535

\item \textbf{Вариант значения:} Числа от 0 до $\times 2^{64}$ - 1 
\end{itemize}

\subsection*{Метод решения}

Моя сортировка реализована в функции TVector<TItem> CountSort(TVector<TItem> \& vec), которая принимает на вход вектор и возвращает его отсортированный вариант. Общий алгоритм поразрядной сортировки основан на том, что мы подсчитываем сколько раз в массиве встречается каждое значение и заполняем массив подсчитанными элементами в соответствующих количествах. Счётчики для всего диапазона чисел создаются зараннее (изначально равны нулю).

Для счётчиков я инциализировал массив для каждого из значений от 0 до максимума среди поданных на вход ключей. Прошелся по элементам массива vec и сопоставил каждому элементу vec индекс массива countArray. Вычислите префикс сумму для каждого индекса массива vec. Создайте массив output одного размера с vec. Прошелся по массиву vec от конца и обновите output[countArray[vec[i]] – 1] = vec[i]. Кроме того, обновил значение countArray[vec[i]] = countArray[vec[i]]-–.



\subsection*{Описание программы}
Были написаны один класс и структура:
\begin{itemize}
\item class Tvector \textbf{--} собственная реализация вектора с произвольным типом
\item struct TItem \textbf{--} структура для хранения пары ключ-значение.
\end{itemize}

Для выполнения задания я использовал:
\begin{itemize}
\item TVector<TItem> vec \textbf{--} вектор, где хранятся пары ключ-значения. 
\end{itemize}
Изначально считываются все входные данные, затем сортируется основной вектор, в конце — вывод отсортированного вектора.

\subsection*{Дневник отладки}
В первый раз был выбран неправильный компилятор в проверяющей системе. Затем проблемой стала синхронизация printf и cin, которая замедляла исполнение программы.
\subsection*{Тест производительности}

Для проверки производительности алгоритма я использовала сравнение со стандартной сортировкой std::sort, сложность которой ${O}({n}\log n)$.
Сравнение производилось на входных данных больших размеров, не превышающих ${5*10^6}$. 

Исходя из графика, представленного ниже, можно увидеть, что сложность моей сортировки подсчетом близка к линейной. А также на таком большом количестве данных сортировка подсчетом оказалась эффективнее стандартной сортировки std::sort.

\begin{figure}[htbp]
    \centering
    \begin{tikzpicture}
        \begin{axis}[
            xlabel={Объём входных данных ($\times 10^6$)},
            ylabel={Время работы (ms)},
            grid=major,
            xmin=0, xmax=5,
            ymin=0, ymax=1400,
            xtick={0,1,2,3,4,5},
            ytick={0,100,200,400,600,800,1000, 1200, 1400},
            legend style={at={(0.5,-0.2)},anchor=north},
            legend columns=2,
            width=0.8\textwidth,
            height=0.5\textwidth,
            ]
            \addplot[color=red,mark=*] coordinates {
                (0,0)
                (1,54.40)
                (2,138.59)
                (3,200.59)
                (4,277.80)
                (5,383.60)
            };
            \addlegendentry{CountSort}
            \addplot[color=green,mark=square] coordinates {
                (0,0)
                (1,309.19)
                (2,636.39)
                (3,962.79)
                (4,1288.79)
                (5,1627.79)
            };
            \addlegendentry{std::sort}
        \end{axis}
    \end{tikzpicture}
    \label{fig:graph}
\end{figure}

\newpage
\subsection*{Выводы}

%Описать область применения реализованного алгоритма. Указать типовые задачи, решаемые им. Оценить сложность программирования, кратко описать возникшие проблемы при решении задачи. 

В результате данной лабораторной работы была написана и отлажена программа на языке C++, осуществляющая ввод пар «ключ-значение», их упорядочивание по возрастанию ключа алгоритмом поразрядной сортировки за линейное время и вывод отсортированной последовательности. Также были написаны аналоги стандартных контейнеров, такого как вектор.

Сложность моей поразрядной сортировки: ${O}$(N+K), где N \textbf{--} количество элементов, а K \textbf{--} максимальное значение ключа. В связи с тем, что максимальное значение ключа ограничено по условию, сложность получается линейной.

Данный алгоритм эффективен для быстрого поиска данных, диапазон ключей которых заранее известен, а также количество элементов сильно превышает количество элементов в массиве. То есть данные содержат много повторяющихся ключей.  

\end{document}
